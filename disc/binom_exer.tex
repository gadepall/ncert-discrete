%\documentclass[journal,12pt,twocolumn]{IEEEtran}
%\usepackage{setspace}
%\usepackage{gensymb}
%\usepackage{caption}
%%\usepackage{multirow}
%%\usepackage{multicolumn}
%%\usepackage{subcaption}
%%\doublespacing
%\singlespacing
%\usepackage{csvsimple}
%\usepackage{amsmath}
%\usepackage{multicol}
%%\usepackage{enumerate}
%\usepackage{amssymb}
%%\usepackage{graphicx}
%\usepackage{newfloat}
%%\usepackage{syntax}
%\usepackage{listings}
%%\usepackage{iithtlc}
%\usepackage{color}
%\usepackage{tikz}
%\usetikzlibrary{shapes,arrows}
%
%
%
%%\usepackage{graphicx}
%%\usepackage{amssymb}
%%\usepackage{relsize}
%%\usepackage[cmex10]{amsmath}
%%\usepackage{mathtools}
%%\usepackage{amsthm}
%%\interdisplaylinepenalty=2500
%%\savesymbol{iint}
%%\usepackage{txfonts}
%%\restoresymbol{TXF}{iint}
%%\usepackage{wasysym}
%\usepackage{amsthm}
%\usepackage{mathrsfs}
%\usepackage{txfonts}
%\usepackage{stfloats}
%\usepackage{cite}
%\usepackage{cases}
%\usepackage{mathtools}
%\usepackage{caption}
%\usepackage{enumerate}	
%\usepackage{enumitem}
%\usepackage{amsmath}
%%\usepackage{xtab}
%\usepackage{longtable}
%\usepackage{multirow}
%%\usepackage{algorithm}
%%\usepackage{algpseudocode}
%\usepackage{enumitem}
%\usepackage{mathtools}
%\usepackage{hyperref}
%%\usepackage[framemethod=tikz]{mdframed}
%\usepackage{listings}
%    %\usepackage[latin1]{inputenc}                                 %%
%    \usepackage{color}                                            %%
%    \usepackage{array}                                            %%
%    \usepackage{longtable}                                        %%
%    \usepackage{calc}                                             %%
%    \usepackage{multirow}                                         %%
%    \usepackage{hhline}                                           %%
%    \usepackage{ifthen}                                           %%
%  %optionally (for landscape tables embedded in another document): %%
%    \usepackage{lscape}     
%
%
%\usepackage{url}
%\def\UrlBreaks{\do\/\do-}
%
%
%%\usepackage{stmaryrd}
%
%
%%\usepackage{wasysym}
%%\newcounter{MYtempeqncnt}
%\DeclareMathOperator*{\Res}{Res}
%%\renewcommand{\baselinestretch}{2}
%\renewcommand\thesection{\arabic{section}}
%\renewcommand\thesubsection{\thesection.\arabic{subsection}}
%\renewcommand\thesubsubsection{\thesubsection.\arabic{subsubsection}}
%
%\renewcommand\thesectiondis{\arabic{section}}
%\renewcommand\thesubsectiondis{\thesectiondis.\arabic{subsection}}
%\renewcommand\thesubsubsectiondis{\thesubsectiondis.\arabic{subsubsection}}
%
%% correct bad hyphenation here
%\hyphenation{op-tical net-works semi-conduc-tor}
%
%%\lstset{
%%language=C,
%%frame=single, 
%%breaklines=true
%%}
%
%%\lstset{
%	%%basicstyle=\small\ttfamily\bfseries,
%	%%numberstyle=\small\ttfamily,
%	%language=Octave,
%	%backgroundcolor=\color{white},
%	%%frame=single,
%	%%keywordstyle=\bfseries,
%	%%breaklines=true,
%	%%showstringspaces=false,
%	%%xleftmargin=-10mm,
%	%%aboveskip=-1mm,
%	%%belowskip=0mm
%%}
%
%%\surroundwithmdframed[width=\columnwidth]{lstlisting}
%\def\inputGnumericTable{}                                 %%
%\lstset{
%%language=C,
%frame=single, 
%breaklines=true,
%columns=fullflexible
%}
% 
%
%\begin{document}
%%
%\tikzstyle{block} = [rectangle, draw,
%    text width=3em, text centered, minimum height=3em]
%\tikzstyle{sum} = [draw, circle, node distance=3cm]
%\tikzstyle{input} = [coordinate]
%\tikzstyle{output} = [coordinate]
%\tikzstyle{pinstyle} = [pin edge={to-,thin,black}]
%
%\theoremstyle{definition}
%\newtheorem{theorem}{Theorem}[section]
%\newtheorem{problem}{Problem}
%\newtheorem{proposition}{Proposition}[section]
%\newtheorem{lemma}{Lemma}[section]
%\newtheorem{corollary}[theorem]{Corollary}
%\newtheorem{example}{Example}[section]
%\newtheorem{definition}{Definition}[section]
%%\newtheorem{algorithm}{Algorithm}[section]
%%\newtheorem{cor}{Corollary}
%\newcommand{\BEQA}{\begin{eqnarray}}
%\newcommand{\EEQA}{\end{eqnarray}}
%\newcommand{\define}{\stackrel{\triangle}{=}}
%
%\bibliographystyle{IEEEtran}
%%\bibliographystyle{ieeetr}
%
%\providecommand{\nCr}[2]{\,^{#1}C_{#2}} % nCr
%\providecommand{\nPr}[2]{\,^{#1}P_{#2}} % nPr
%\providecommand{\mbf}{\mathbf}
%\providecommand{\pr}[1]{\ensuremath{\Pr\left(#1\right)}}
%\providecommand{\qfunc}[1]{\ensuremath{Q\left(#1\right)}}
%\providecommand{\sbrak}[1]{\ensuremath{{}\left[#1\right]}}
%\providecommand{\lsbrak}[1]{\ensuremath{{}\left[#1\right.}}
%\providecommand{\rsbrak}[1]{\ensuremath{{}\left.#1\right]}}
%\providecommand{\brak}[1]{\ensuremath{\left(#1\right)}}
%\providecommand{\lbrak}[1]{\ensuremath{\left(#1\right.}}
%\providecommand{\rbrak}[1]{\ensuremath{\left.#1\right)}}
%\providecommand{\cbrak}[1]{\ensuremath{\left\{#1\right\}}}
%\providecommand{\lcbrak}[1]{\ensuremath{\left\{#1\right.}}
%\providecommand{\rcbrak}[1]{\ensuremath{\left.#1\right\}}}
%\theoremstyle{remark}
%\newtheorem{rem}{Remark}
%\newcommand{\sgn}{\mathop{\mathrm{sgn}}}
%\providecommand{\abs}[1]{\left\vert#1\right\vert}
%\providecommand{\res}[1]{\Res\displaylimits_{#1}} 
%\providecommand{\norm}[1]{\left\Vert#1\right\Vert}
%\providecommand{\mtx}[1]{\mathbf{#1}}
%\providecommand{\mean}[1]{E\left[ #1 \right]}
%\providecommand{\fourier}{\overset{\mathcal{F}}{ \rightleftharpoons}}
%%\providecommand{\hilbert}{\overset{\mathcal{H}}{ \rightleftharpoons}}
%\providecommand{\system}{\overset{\mathcal{H}}{ \longleftrightarrow}}
%	%\newcommand{\solution}[2]{\textbf{Solution:}{#1}}
%\newcommand{\solution}{\noindent \textbf{Solution: }}
%\newcommand{\myvec}[1]{\ensuremath{\begin{pmatrix}#1\end{pmatrix}}}
%\providecommand{\dec}[2]{\ensuremath{\overset{#1}{\underset{#2}{\gtrless}}}}
%\DeclarePairedDelimiter{\ceil}{\lceil}{\rceil}
%%\numberwithin{equation}{section}
%%\numberwithin{problem}{subsection}
%%\numberwithin{definition}{subsection}
%\makeatletter
%\@addtoreset{figure}{section}
%\makeatother
%
%\let\StandardTheFigure\thefigure
%%\renewcommand{\thefigure}{\theproblem.\arabic{figure}}
%\renewcommand{\thefigure}{\thesection}
%
%
%%\numberwithin{figure}{subsection}
%
%%\numberwithin{equation}{subsection}
%%\numberwithin{equation}{section}
%%\numberwithin{equation}{problem}
%%\numberwithin{problem}{subsection}
%\numberwithin{problem}{section}
%%%\numberwithin{definition}{subsection}
%%\makeatletter
%%\@addtoreset{figure}{problem}
%%\makeatother
%\makeatletter
%\@addtoreset{table}{section}
%\makeatother
%
%\let\StandardTheFigure\thefigure
%\let\StandardTheTable\thetable
%\let\vec\mathbf
%%%\renewcommand{\thefigure}{\theproblem.\arabic{figure}}
%%\renewcommand{\thefigure}{\theproblem}
%
%%%\numberwithin{figure}{section}
%
%%%\numberwithin{figure}{subsection}
%
%
%
%\def\putbox#1#2#3{\makebox[0in][l]{\makebox[#1][l]{}\raisebox{\baselineskip}[0in][0in]{\raisebox{#2}[0in][0in]{#3}}}}
%     \def\rightbox#1{\makebox[0in][r]{#1}}
%     \def\centbox#1{\makebox[0in]{#1}}
%     \def\topbox#1{\raisebox{-\baselineskip}[0in][0in]{#1}}
%     \def\midbox#1{\raisebox{-0.5\baselineskip}[0in][0in]{#1}}
%
%\vspace{3cm}
%
%\title{ 
%%	\logo{
%BINOMIAL THEOREM
%%	}
%}
%
%\author{ G V V Sharma$^{*}$% <-this % stops a space
%	\thanks{*The author is with the Department
%		of Electrical Engineering, Indian Institute of Technology, Hyderabad
%		502285 India e-mail:  gadepall@iith.ac.in. All content in this manual is released under GNU GPL.  Free and open source.}
%	
%}	
%
%\maketitle
%
%%\tableofcontents
%
%\bigskip
%
%\renewcommand{\thefigure}{\theenumi}
%\renewcommand{\thetable}{\theenumi}
%\textbf{Expand each of the expression in the question number 1 to 5 }
%
%
%\begin{enumerate}[label=\arabic*]
%\numberwithin{equation}{enumi}
%
%
\renewcommand{\theequation}{\theenumi}
\begin{enumerate}[label=\arabic*.,ref=\thesubsection.\theenumi]
\numberwithin{equation}{enumi}
\item $(1-2x)^5$\\
\item $(\frac{2}{x} - \frac{x}{2})^5$\\
\item $(2x - 3)^6$\\
\item $(\frac{x}{3} + \frac{1}{x})^5$\\
\item $(x + \frac{1}{x})^6$\\
\\


\textbf{Using binomial theorem,evaluate each of the following:}\\

\item $(96)^3$\\
\item $(102)^5$\\
\item $(101)^4$\\
\item $(99)^5$\\

\item Using Binomial Theorem ,indicate which number is larger $(1.1)^{10000}$ or 1000.\\
\item Find $(a+b)^4 - (a-b)^4.$ Hence evaluate \\
$(\sqrt{3} + \sqrt{2})^4$ - $(\sqrt{3} - \sqrt{2})^4.$\\
\item Find $(x + 1)^6 + (x - 1)^6.$ Hence or otherwise evaluate $(\sqrt{2} + 1)^6 + (\sqrt{2} - 1)^6.$\\
\item Show that $9^{n+1} - 8n - 9$ is divisible by 64, whenever n is a positive integer.\\
\item Prove that $ \sum \limits_{r=0}^{n}\ 3^r$ $ ^nC_r\ = 4 ^n$.\\






\textbf{Find the coefficient of}\\


\item $x^5$ in $(x+3)^8$\\
\item $a^5b^7$ in $(a - 2b)^{12}.$\\

\textbf {Write the general term in the expansion of}\\
\item $(x^2 - y)^6$\\
\item $(x^2 - yx)^{12}, x \neq 0.$\\
\item Find the $4^{th}$ term in the expansion of $(x - 2y)^{12}.$\\
\item Find the $13^{th}$ term in the expansion of $ (9x - \frac{1}{3\sqrt{x}})^{18}$,$ x \neq 0.$\\

\textbf{Find the middle terms in the expansions of }\\
\item $(3 - \frac{x^3}{6})^7$\\
\item $(\frac{x}{3} + 9y)^{10}$\\
\item In the expansion of $(1 + a)^{m+n}$,prove that coefficients of $a^m$ and $a^n$ are equal.\\
\item The coefficients of the $(r - 1)^{th},r^{th}$ and $(r + 1)^{th}$ terms in the expansion of $(x + 1)^n$ are in the ratio 1 : 3: 5.Find n and r.\\
\item Prove that the coefficient of $x^n$ in the expansion of $(1 + x)^{2n}$ is twice the coefficient of $x^n$ in the expansion of $(1 + x)^{2n - 1}.$\\
\item Find the positive value of m for which the coefficient of $x^2$ in the expansion $(1 + x)^m$ is 6.\\
\item Find a,b and n in the expansion of $(a + b)^n$ if the first three terms of the expansion are 729,7290 and 30375,respectively.\\
\item Find the coefficient of $x^5$ in the product $(1 + 2x)^6 (1 - x)^7$ using binomial theorem.\\
\item Find a if the coefficients of $x^2$ and $x^3$ in the expansion of $(3 + ax)^9$ are equal.\\
\item If a and b are distinct integers, prove that a - b is a factor of $a^n - b^n$,whenever n is a positive integer.\\
\textbf{Hint} write $a^n = (a - b + b)^n$ and expand\\
\item Evaluate $(\sqrt{3} + \sqrt{2})^6$ - $(\sqrt{3} - \sqrt{2})^6$\\
\item Find the value of $(a^2 + \sqrt{a^2 - 1})^4$ + $(a^2 - \sqrt{a^2 - 1})^4.$\\
\item Find an approximation of $(0.99)^5$ using the first three terms of its expansion.\\
\item Find n, if the ratio of the fifth term from the beginning to the fifth term from the end in the expansion of $(\sqrt[4]{2} + \frac{1}{\sqrt[4]{3}})^n$ is $\sqrt{6} : 1$\\
\item Expand using BinomialTheorem $(1 + \frac{x}{2} - \frac{2}{x})^4$, $x \neq 0.$\\
\item Find the expansion of $(3x^2 - 2ax + 3a^2)^3$ using binomial theorem.




\end{enumerate}
%\end{document}
